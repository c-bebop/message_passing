\documentclass[xcolor=dvipsnames]{article}

% Bibtex
\usepackage{cite, authordate1-4}
\usepackage{url}

% Open Sans Package!
\usepackage[default,scale=0.95]{opensans}
\usepackage[T1]{fontenc}

\usepackage{transparent}

\usepackage[dvipsnames]{xcolor}
\definecolor{blue}{cmyk}{0.81,0.26,0,0.48}
\definecolor{code_backgrond}{cmyk}{0.02,0.01,0,0.07}
\definecolor{red}{cmyk}{0,0.76,0.76,0.48}
\definecolor{lbcolor}{rgb}{1,1,1}
\definecolor{mygray}{rgb}{0.3,0.3,0.3}

% GLOSSAR BEGIN
\usepackage[acronym]{glossaries}

% Listing for Code
\usepackage{listings}


\lstset{ %
	language=Erlang,                % choose the language of the code
	basicstyle=\ttfamily\footnotesize\color{black},       % the size of the fonts that are used for the code
	numbers=left,                   % where to put the line-numbers
	numberstyle=\color{black}\footnotesize,      % the size of the fonts that are used for the line-numbers
	stepnumber=1,                   % the step between two line-numbers. If it is 1 each line will be numbered
	numbersep=3pt,                  % how far the line-numbers are from the code
	showspaces=false,               % show spaces adding particular underscores
	showstringspaces=false,         % underline spaces within strings
	showtabs=false,                 % show tabs within strings adding particular underscores
	%frame=single,           % adds a frame around the code
	tabsize=2,          % sets default tabsize to 2 spaces
	captionpos=b,           % sets the caption-position to bottom
	breaklines=true,        % sets automatic line breaking
	breakatwhitespace=false,    % sets if automatic breaks should only happen at whitespace
	escapeinside={\%*}{*)},         % if you want to add a comment within your code
	keywordstyle=\color{red},
	commentstyle=\color{mygray},
	stringstyle=\color{black},
	backgroundcolor=\color{code_backgrond}
}

% Generate the glossary
\makenoidxglossaries

%Term definitions
\newacronym{os}{OS}{Operating System}
\newacronym{mpi}{MPI}{Message-Passing Interface}
\newacronym{rpc}{RPC}{Remote Procedure Calls}
\newacronym{tcp}{TCP/IP}{Transmission Control Protocol / Internet Protocol}
\newacronym{xml}{XML}{Extensible Markup Language}
\newacronym{json}{JSON}{JavaScript Object Notation}
\newacronym{ipc}{IPC}{Inter-Process Communication}
\newacronym{rtp}{RTP}{Real-Time Transport Protocol}
\newacronym{udp}{UDP}{User Datagram Protocol}
\newacronym{osi}{OSI}{Open Systems Interconnection Reference Model}
\newacronym{iso}{ISO}{International Organization for Standardization}
\newacronym{fifo}{FIFO}{First In - First Out}
\newacronym{pid}{PID}{Process Identifier}
\newacronym{mpif}{MPIF}{Message-Passing Interface Forum}
\newacronym{hpc}{HPC}{High-Performance Computing}

% GLOSSAR END

% For HTML conversion
%\usepackage{tex4ht}

\title{\scshape{\textbf{\textcolor{blue}{Introductory Guide to Message Passing}}\\In Distributed Systems}}

\author{Florian Willich \\ Hochschule f\"ur Technik und Wirtschaft Berlin \\ University of Applied Sciences Berlin \\ Course: Distributed Systems \\ Lecturer: Prof. Dr. Christin Schmidt}
        
\date{\today}

\begin{document}

\maketitle

\begin{abstract}

Message passing in distributed systems is a model to exchange messages within a process pair by making use of several standards and implementation details. Those have been developed to offer the right message passing models for the different areas of applications. The programming language Erlang natively supports an asynchronous message passing model which makes the implementation of concurrent applications transparent to the software developer.

\end{abstract}

\tableofcontents

\newpage

\section{\scshape{\textcolor{blue}{Message Passing}}} \label{introduction}

\subsection{\scshape{\textcolor{blue}{Introduction}}}

To provide services and execute tasks, a distributed system has to have a proficient communication model implementation. There are several models of communication in distributed systems. With this paper I am giving an introduction to different \textit{message-oriented message passing communication models} \cite[chap. 4.3 on p. 140 - 141]{tanenbaum}.\\

\noindent As human beings we already have a deep understanding of different models of message passing. While you read these words, I am passing a message to you, which seems to be a trivial thing to people who can read. In more philosophical terms, now that you are reading those words, a message is being passed from one entity (me, the writer) to another entity (you, the reader) which is obviously non trivial considering all the assumptions we would have to make to actually realize this message passing model between human beings.\\

\noindent Message passing in distributed systems is based on messages, composed of bit strings, exchanged within a process pair which would be the equivalent to the entity pair. It is important to understand that message passing is a model designed for \gls{ipc}. Whether those processes are located on one or on two systems, is irrelevant to the provided functionality \cite[ch. 4 - 4.1.1 on p. 115 - 117]{tanenbaum}.

\subsection{\scshape{\textcolor{blue}{Requirements in Theory}}}

\noindent The following items are the theoretical basic requirements for a message passing model:

\begin{itemize}

\item \textbf{Connectivity}: A connection to communicate has to be established between the process pair
\item \textbf{Ability}: Each process has to be able to receive or send messages
\item \textbf{Integrity}: Sent messages have to be delivered as is
\item \textbf{Intelligibility}: The receiving process has to be able to interpret the message as intended

\end{itemize}

\noindent I intentionally left out the requirement of executability without which there is no effect when the message is passed. It is an implementation detail of the executed program code to call the desired instructions in the receiving process, and thus not a requirement of the message passing model in itself.

\subsection{\scshape{\textcolor{blue}{Requirements Provider in Practice}}}

\noindent To provide the requirements mentioned above several message passing models resulting in well defined standards and concrete implementations have been developed in the last decades. To facilitate understanding the relations between the theoretical requirements and a concrete implementation, I chose the \gls{tcp} which uses several socket primitives as example \cite[chap 4.3.1 on p. 141 - 142]{tanenbaum}:

\begin{itemize}

\item \textbf{Connectivity}: \gls{tcp} provides the \textit{Socket} primitive creating a socket end point and also the \textit{Bind} primitive to bind a local address to that socket. The \textit{Connect} primitive then provides the functionality to establish a connection.

\item \textbf{Ability}: \gls{tcp} provides the two primitives \textit{Send} and \textit{Receive} to simply send and receive data via the connection.

\item \textbf{Integrity}: \gls{tcp} provides several mechanisms to ensure that there has been no data loss when sending or receiving messages. On the other hand this makes the protocol slower than other protocols such as the \gls{rtp} or the \gls{udp}.

\end{itemize}

\noindent The requirements of \textbf{Intelligibility} do not fall within the responsibility of \gls{tcp} and thus other standards have to take place, such as \gls{xml} or \gls{json} to structure the messages in a standardized manner.\\

\noindent There are other socket primitives than the above mentioned to provide the functionality of \gls{tcp} \cite[ch. 4.3.1 on p. 141]{tanenbaum}. The referenced book \cite{tanenbaum} provides useful and valuable information on this topic.\\

\noindent It is also important to understand that this was just a round-up of how one can make use of the socket primitives to implement message passing in an application. There is plenty to discuss about how message passing is realized, such as the \gls{osi} which was developed by the \gls{iso} to model the different layers of network oriented communication \cite[ch. 4.1.1 on p. 116]{tanenbaum}. Furthermore, an important detail is that the \gls{os} has to reserve local memory to provide a buffer for the incoming and outgoing messages.

\subsection{\scshape{\textcolor{blue}{Asynchronous vs. Synchronous Message Passing}}}

\noindent After discussing a protocol to handle the socket primitives, it is necessary to consider one major difference in message passing models, which is whether to use asynchronous or synchronous message passing. Passing a message synchronously means that the called \textit{message send routine} returns after the request has been successfully transmitted. Receiving a message synchronously means that the called message receiving routine reads a specific amount of bytes from the socket and returns that message.\\

\noindent To pass or receive a message asynchronously, an additional application layer i.e. middleware is introduced. The call of a send routine can either return when the middleware took over the transmission of the request, or when it successfully sent the request to the receiver, or when it successfully sent the request to the receiver assuring this by a corresponding message \cite[ch. 4.1 on p. 125]{tanenbaum}.\\

\noindent The middleware can also perform some preparatory work e.g. separating stand-alone request bit strings and parsing them into a data structure and returning it when calling the receive routine of the message passing middleware.\\

\noindent One key model to provide such middleware facilities are message queues. A message queue is a data structure to store messages in by the \gls{fifo} principle. This enables the application to asynchronously send and receive messages by offering an incoming message queue and an outgoing message queue. This makes all of the send and receive mechanisms fully transparent to the application \cite[ch. 4.3.2 on p. 145 - 147]{tanenbaum}.

\section{\scshape{\textcolor{blue}{Message-Passing Interface Standard}}} \label{message_passing_interface}

Passing messages between processes adds significant overhead to the communication model. Nevertheless, providing a well defined interface to control exactly how messages are passed regains the ability to write \gls{hpc} applications.\\

\noindent The \gls{mpi} is a message-passing library interface specification, created for high performance and scalable distributed systems where high-efficiency is needed.  The \gls{mpi} standard was designed by the \sloppy \gls{mpif} which is an open group with representatives from many organizations. The current version of the \gls{mpi} standard is MPI-3.0 \cite[ch. Abstract/ii \& ch. Acknowledgements/xx \& ch. 1.1 on p. 1 \& ch. 1.2 on p. 2]{mpi}.\\

\noindent The \gls{mpif} aims to offer a standard which establishes a practical, portable, efficient and flexible way of implementing message-passing in various high-level programming languages (e.g. C, C++, Fortran) \cite[ch. History/iii]{mpi}. Furthermore, the \gls{mpi} simplifies the communication primitives and brings them to an abstraction level to perfectly fit the programmer's needs of writing efficient and clean code for such \gls{hpc} distributed systems \cite[ch. 4.3.1 on p 143]{tanenbaum}.\\

\noindent \gls{tcp} does not fit those requirements. While the socket primitives \textit{read} and \textit{write} are sufficient for several general-purpose protocols (managing the communication across networks), they are insufficient for high-speed interconnection networks such as super computers or server clusters. The \gls{mpi} standard offers a set of functions and datatypes with which the software developer is able to explicitly execute synchronous and asynchronous message passing routines \cite[ch. 4.3 on p. 143]{tanenbaum}.\\

\noindent The \gls{mpif} offers a very detailed technical report \cite{mpi} of the \gls{mpi} standard which is not only defining the standard but offers detailed information on the organization and motivation.

\section{\scshape{\textcolor{blue}{Definition}}}

The given task of Prof. Dr. Christin Schmidt was to write a technical report defining the term \textit{Message Passing}. After discussing the theoretical basis and practical implementations, message passing in distributed systems appears to be a technical term to characterize a communication model in distributed systems that deals with messages:

\begin{quote}
\textit{Message passing in distributed systems is a model to exchange messages within a process pair by making use of several standards and implementation details. The specifically used message passing model can diverge extremely in its provided facilities and defines how to establish the connection, send and receive messages.}
\end{quote}

\noindent However, no message passing model defines the medium that transports the messages nor the outcome of a message and thus the following delimitation has to be made:

\begin{quote}
\textit{The physical conditions and the implementation of executing the desired instructions is not defined by any message passing model.}
\end{quote}

\section{\scshape{\textcolor{blue}{Message Passing in Erlang}}} \label{erlang}

The preceding discourse repeatedly demonstrated that message passing does not only mean adding computational overhead but also a number of things the programmer has to keep track of. The message passing model implemented in Erlang simplifies and abstracts the use of message passing which enables the programmer to keep the focus on application logic. Erlang's transparent message passing model is therefore well suited as an illustration.

\subsection{\scshape{\textcolor{blue}{Introduction}}}

Erlang is a functional, declarative programming language written for the need of real-time, non-stop, concurrent, very large and distributed system applications \cite[chap. 1 / p. 1]{Armstrong96erlang}. While the language was originally designed by Joe Armstrong, Robert Virding and Mike Williams for Ericsson (a Swedish telecommunications provider) in 1986, it finally became open source in 1998, thanks to the open source initiatives lead by Linux \cite[chap. 8 on p. 39]{erl_history}.\\

\noindent Joe Armstrong, who started to design Erlang by adding functionality to Prolog, named the new programming language after the Danish mathematician \sloppy Agner \sloppy Krarup  Erlang (creator of the Erlang loss formula) following the tradition of naming programming languages after dead mathematicians \cite[chap. 4.1 on p. 13]{erl_history}.\\

\noindent Erlang uses a native, asynchronous message passing model to communicate between light weight processes, also called actors \cite[chap. 1 on p. 1]{Armstrong96erlang}. Erlang does not make heavy use of the executing \gls{os} to provide the described concurrency model, which implicitly means that Erlang decouples the underlying \gls{os} and thus provides a cross-platform and transparent message passing model \cite[chap. 1  \& 3 on p. 1 - 3]{Armstrong96erlang}.

\subsection{\scshape{\textcolor{blue}{Concurrency in Erlang}}}

\noindent Erlang provides semantics and built-in functions to parallelize  applications by message passing \cite[ch. 4.3 on p. 95 - 104]{erl_doc}:

\begin{itemize}

\item \textcolor{red}{spawn}(\textit{Module}, \textit{Exported Function}, \textit{List of Arguments})\\
A function which creates a new actor by running the \textit{Exported Function} with the \textit{List of Arguments} located in the \textit{Module} (a set of functions located in one file) returning the \gls{pid}, which uniquely identifies the created actor for addressing.

\item The \textcolor{red}{receive} construct allows the function executed by an actor to receive messages by using a message queue.

\item The \textcolor{red}{!} operator sends the right-handed term to the \sloppy left-handed \gls{pid}. The right-handed term is the sent message.

\item \textcolor{red}{self}() \\
A function returning the \gls{pid} of the actor executing the function.

\end{itemize}

\noindent Although the semantics and functions described above may require some foreknowledge on the Erlang programming language, it is important to point out that Erlang is not only offering an easily intelligible interface for concurrency, it rather is a concurrent functional programming language. Nevertheless, there is plenty to discuss about the concurrency model of Erlang e.g. how to manage processes or handle errors. It is recommended to take a look at the referenced official Erlang documentation \cite{erl_doc} or to \textit{learn you some Erlang} on \url{www.learnyousomeerlang.com}.

\subsection{\scshape{\textcolor{blue}{Implementation of a Simple Diffie-Hellman Key Exchange Algorithm}}} \label{erlang_implementation}

The following implementation of a Diffie-Hellman key exchange algorithm in Erlang is simplified. Please note that this Erlang application is not in any way acceptable as an adequate implementation of the Diffie-Hellman key exchange algorithm for use in the field! The code demonstrates the simplicity of implementing concurrent applications in Erlang by using message passing. Due to the need for a pow function returning the data type \textit{Integer} I implemented my own pow function, see Appendix \ref{my_math}. The sourcecode is hosted by Github:

\begin{center}
\url{https://github.com/c-bebop/message_passing}
\end{center}

\noindent Licensed under the MIT License. You're welcome to contribute!

\lstinputlisting{../erlang_diffie_hellman/src/diffie_hellman.erl}

\subsection{\scshape{\textcolor{blue}{Running the Application}}}

To run the application (\ref{erlang_implementation}) in a Linux shell one shall do the following (assuming that Erlang is already installed, help can be found at \url{http://www.erlang.org/doc/installation_guide/INSTALL.html}).

\subsubsection{Preparatory Work}

\noindent Open a Linux shell, go to the \textit{source} directory of the Erlang code and type:

\begin{lstlisting}[language=bash, numbers=none]
$ erl -sname bob
\end{lstlisting}

\noindent This will output (meta data depends of the executing system):

\begin{lstlisting}[language=bash]
Erlang/OTP 17 [erts-6.3] [source] [64-bit] [smp:4:4] [async-threads:10] [hipe] [kernel-poll:false]

Eshell V6.3  (abort with ^G)
(bob@localhost)1> 
\end{lstlisting}

\noindent The option \textit{-sname bob} tells the Erlang shell to run on a node called \textit{bob}. Now compile the two provided modules as follows:

\begin{lstlisting}[language=bash]
(bob@localhost)1> c(diffie_hellman).
{ok,diffie_hellman}
(bob@localhost)2> c(my_math).
{ok,my_math}
\end{lstlisting}

\subsubsection{On One Node}

To run the application on one node execute the \textit{startExample} function as follows:

\begin{lstlisting}[language=bash]
(bob@localhost)3> diffie_hellman:startExample().
"Alice" (<0.50.0>): The shared private Key, exchanged with "Bob" is: 2
"Bob" (<0.51.0>): The shared private Key, exchanged with "Alice" is: 2
{<0.50.0>,<0.51.0>}
\end{lstlisting}

\noindent Executing the function \textit{startExample} spawns a process that executes the \textit{listenKeyExchange} function with \textit{15} as the \textit{private key} and \textit{"Alice"} as the name and binds the returning \gls{pid} to the value called Alice. Afterwards, \textit{startExample} spawns another process that executes the \textit{startKeyExchange} function with \textit{23} for \textit{P}, \textit{5} for \textit{G}, \textit{6} for the \textit{private key}, the \gls{pid} of Alice and \textit{"Bob"} as the name which binds the returning \gls{pid} to the value called Bob. Eventually the function returns the two pids of \textit{Alice} and \textit{Bob}. The produced output indicates that Alice has the \gls{pid} <0.50.0> and the calculated private key exchanged with Bob is 2. Bob has the \gls{pid} <0.51.0> and computed the same private key (please note that the returning \gls{pid} is not determined).

\subsubsection{On Two Nodes}

\noindent To run the application on two nodes open another shell, go to the \textit{source} directory of the Erlang code and type:

\begin{lstlisting}[language=bash, numbers=none]
$ erl -sname alice
\end{lstlisting}

\noindent This will output (meta data depends of the executing system):

\begin{lstlisting}[language=bash]
Erlang/OTP 17 [erts-6.3] [source] [64-bit] [smp:4:4] [async-threads:10] [hipe] [kernel-poll:false]

Eshell V6.3  (abort with ^G)
(alice@localhost)1>
\end{lstlisting}

\noindent Now another Erlang shell runs on the node \textit{alice} where the source files shall be compiled, too. Switching to the Erlang shell on node \textit{bob} and executing the function \textit{startRemoteExample} with the atom \textit{alice@localhost} as the transferred parameter as follows:

\begin{lstlisting}[language=bash]
(bob@localhost)4> diffie_hellman:startRemoteExample(alice@localhost).
\end{lstlisting}

\noindent prints out the following:

\begin{lstlisting}[language=bash]
{<9879.54.0>,<0.58.0>}
"Alice" (<9879.54.0>): The shared private Key, exchanged with "Bob" is: 2
"Bob" (<0.58.0>): The shared private Key, exchanged with "Alice" is: 2
\end{lstlisting}

\noindent The atom \textit{alice@localhost} specifies on which node the actor shall be spawned that executes the \textit{listenKeyExchange} function

\noindent Now the \textit{Alice} actor is located on the node \textit{alice}. The output is still printed out in the \textit{bob} node, since the Erlang io system recognizes where the process is spawned from and sends all the output to it \cite[ch. 4.3.4 on p. 104]{erl_doc}.

\section{\scshape{\textcolor{blue}{Personal Closing Remarks}}}

\noindent With this technical report I do not presume describing message passing in its wholeness. Nevertheless, my goal was to introduce the reader to this field hoping to motivate implementing one's own applications that use message passing, as well as gaining deeper knowledge in this field. For instance, it remains to be discussed how the \gls{os} is involved in the process of message passing, how a message should be composed in the light of the occurrence of computational overhead and how to better manage concurrent applications in Erlang. This could be the subject-matter of subsequent technical reports.

\newpage 

\begin{appendix}

\section{my\_math Erlang Module}\label{my_math}

\lstinputlisting{../erlang_diffie_hellman/src/my_math.erl}

\end{appendix}

\printnoidxglossaries

\newpage

\bibliographystyle{authordate1}
\bibliography{lib}

\vfill
\begin{center}
This document was written with \LaTeX 
\\Typeface: Open Sans by Steve Matteson.
\end{center}

% To compile bibtex and latex manual:
% bibtex belegarbeit.aux
% latex belegarbeit.tex

\end{document}