\documentclass[aspectratio=169]{beamer}

% This document was written with the help of
% http://en.wikibooks.org/wiki/LaTeX/Presentations

% The theme of this presentation
\usetheme{default}

% Bibtex
\usepackage{cite, authordate1-4}

% Graphics package
\usepackage{graphicx}

\definecolor{blue}{cmyk}{0.81,0.26,0,0.48}
\definecolor{red}{cmyk}{0,0.76,0.76,0.48}

\setbeamercolor{title}{fg=white}
\setbeamercolor{titlelike}{fg=white}
\setbeamercolor{structure}{bg=blue}
\setbeamercolor{itemize item}{fg=blue}

\hypersetup{
colorlinks=true,
linkcolor=blue
}

% Makes more space between new lines
\setlength{\parskip}{10pt plus 1pt minus 1pt}

% Open Sans Package
\usepackage[default,scale=0.95]{opensans}
\usepackage[T1]{fontenc}

% Title page
\title{\scshape{\textbf{Introductory Guide to Message Passing}\\In Distributed Systems}}
\author[Florian Willich]{Florian Willich}
\institute[BIT]
{
  Hochschule f\"ur Technik und Wirtschaft Berlin\\
  University of Applied Sciences Berlin\\
  Course: Distributed Systems\\ 
  Lecturer: Prof. Dr. Christin Schmidt
}
\date{\today}

\begin{document}

\frame{\titlepage}

\begin{frame}
\frametitle{Table of Contents}
\tableofcontents
\end{frame}

\section{Introduction}
\begin{frame}

\frametitle{Introduction}

\begin{center}
What is message passing?\\
\pause
\vspace{6pt}
What is message passing in distributed systems? \\
\end{center}

\begin{flushright}
\cite{tanenbaum}
\end{flushright}

\end{frame}

\section{Requirements in Theory}
\begin{frame}

\frametitle{Requirements in Theory}

\begin{itemize}
\item Connectivity
\pause
\item Ability
\pause
\item Integrity
\pause
\item Intelligibility
\end{itemize}

\pause
Executability is not part of the message passing model!


\end{frame}

\section{Requirements Provider in Practice}
\begin{frame}

\frametitle{Requirements Provider in Practice}

TCP/IP provides several \textit{socket primitives}:

\begin{itemize}
\item Connectivity: \textit{Socket} and \textit{Bind}
\pause
\item Ability: \textit{Send} and \textit{Receive}
\pause
\item Integrity: Several mechanisms provided
\pause
\item Intelligibility: Do not fall within the responsibility of TCP/IP
\end{itemize}

\pause
There is still plenty to discuss! \cite{tanenbaum}
\end{frame}

\section{Asynchronous vs. Synchronous Message Passing}
\begin{frame}

\frametitle{Asynchronous vs. Synchronous Message Passing}

\begin{itemize}
\item Synchronous: \\
Communication primitives called directly
\pause
\item Asynchronous: \\
Middleware takes place (Message Queues)
\end{itemize}

\begin{flushright}
\cite{tanenbaum}
\end{flushright}

\end{frame}

\section{Message-Passing Interface Standard}

\begin{frame}

\frametitle{Message-Passing Interface Standard}

\begin{itemize}
\item designed by the Message-Passing Interface Forum
\pause
\item defines a set of functions and datatypes for message passing
\pause
\item aims High-Performance Computing Distributed Systems
\end{itemize}

\begin{flushright}
\cite{mpi}\\
\cite{tanenbaum}
\end{flushright}

\end{frame}

\section{Definition}

\begin{frame}

\frametitle{Definition}

\begin{center}
\textit{Message passing in distributed systems is a model to exchange messages within a process pair by making use of several standards and implementation details. The specifically used message passing model can diverge extremely in its provided facilities and defines how to establish the connection, send and receive messages.}
\end{center}
\pause
\begin{center}
\textit{The physical conditions and the implementation of executing the desired instructions is not defined by any message passing model.}
\end{center}

\end{frame}

\section{Erlang Programming Language}

\begin{frame}

\frametitle{Erlang Programming Language}

\begin{itemize}
\item functional
\pause
\item declarative
\pause
\item for real-time, non-stop, concurrent, very large and distributed system applications
\end{itemize}

\begin{flushright}
\cite{Armstrong96erlang}\\
\cite{erl_history}
\end{flushright}

\end{frame}

\begin{frame}

\frametitle{Concurrency in Erlang}

\begin{itemize}

\item Function \textcolor{red}{spawn}(\textit{Module}, \textit{Exported Function}, \textit{List of Arguments})
\pause
\item Construct \textcolor{red}{receive}
\pause
\item operator \textcolor{red}{!}
\pause
\item Function \textcolor{red}{self}()

\end{itemize}

\begin{flushright}
\cite{erl_doc}\\
\cite{erl}
\end{flushright}

\end{frame}

\section{Live Demo}

\begin{frame}

\frametitle{Live Demo}

\begin{center}
Diffie-Hellman Key Exchange Algorithm implemented in Erlang
\end{center}

\end{frame}

\section{Q\&A}
\begin{frame}

\frametitle{Q\&A}

\begin{center}
Thank you for your kind attention!\\
Any Questions?

\vfill
This document was written with \LaTeX 
\\Typeface: Open Sans by Steve Matteson.
\end{center}

\end{frame}

\begin{frame}

\frametitle{References}

\begin{tiny}

\bibliographystyle{authordate1}
\bibliography{lib}

\end{tiny}

\end{frame}

% etc
\end{document}