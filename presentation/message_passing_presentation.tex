\documentclass[aspectratio=169]{beamer}

% This document was written with the help of
% http://en.wikibooks.org/wiki/LaTeX/Presentations

% The theme of this presentation
\usetheme{default}

\usepackage{german}

% Bibtex
\usepackage{cite, authordate1-4}

% Graphics package
\usepackage{graphicx}

\definecolor{blue}{cmyk}{0.81,0.26,0,0.48}
\definecolor{red}{cmyk}{0,0.76,0.76,0.48}

\setbeamercolor{title}{fg=white}
\setbeamercolor{titlelike}{fg=white}
\setbeamercolor{structure}{bg=blue}
\setbeamercolor{itemize item}{fg=blue}

\addtobeamertemplate{navigation symbols}{}{%
    \usebeamerfont{footline}%
    \usebeamercolor[fg]{footline}%
    \hspace{2em}%
    \insertframenumber/\inserttotalframenumber
}

\hypersetup{
colorlinks=true,
linkcolor=blue
}

% Makes more space between new lines
\setlength{\parskip}{10pt plus 1pt minus 1pt}

% Open Sans Package
\usepackage[default,scale=0.95]{opensans}
\usepackage[T1]{fontenc}

% Title page
\title{\scshape{\textbf{Einf"uhrung in Message Passing}\\in verteilten Systemen}}
\author[Florian Willich]{Florian Willich}
\institute[BIT]
{
  Hochschule f"ur Technik und Wirtschaft Berlin\\
  Kurs: Verteilte Systeme\\ 
  Dozentin: Prof. Dr. Christin Schmidt
}
\date{\today}

\begin{document}

\frame{\titlepage}

\begin{frame}
\frametitle{Table of Contents}
\tableofcontents
\end{frame}

\section{Einf"uhrung}
\begin{frame}

\frametitle{Einf"uhrung}

\begin{center}
Was ist Message Passing?\\
\pause
\vspace{6pt}
Was ist Message Passing in verteilten Systemen? \\
\end{center}

\begin{flushright}
\cite{tanenbaum}
\end{flushright}

\end{frame}

\section{theoretische Anforderungen}
\begin{frame}

\frametitle{theoretische Anforderungen}

\begin{itemize}
\item Verbindung
\pause
\item Bef"ahigung
\pause
\item Vollst"andigkeit
\pause
\item Verst"andlichkeit
\end{itemize}

\pause
Ausf"uhrbarkeit ist nicht teil des Message Passing Modells!

\end{frame}

\section{synchroner und asynchroner Nachrichtenaustausch}
\begin{frame}

\frametitle{synchroner und asynchroner Message Passing}

\begin{itemize}
\item \textbf{synchron} \\
Kommunikations-Primitive werden direkt aufgerufen.
\pause
\item \textbf{asynchron} \\
Eine Dienstschicht (Middleware) wird eingef"uhrt (z.B. Message Queues).
\end{itemize}

\begin{flushright}
\cite{tanenbaum}
\end{flushright}

\end{frame}

\section{Definition}

\begin{frame}

\frametitle{Definition}

\begin{center}
\textit{Message Passing in verteilten Systemen ist ein Modell um Nachrichten innerhalb eines Prozesspaares auszutauschen. Dabei werden verschiedene Standards und Implementierungsdetails zur Hilfe genommen. Das speziell zum Einsatz kommende Modell kann sich extrem von anderen hinsichtlich der bereitgestellten Funktionalit"at unterscheiden und definiert wie die Verbindung hergestellt und Nachrichten gesendet und empfangen werden.}
\end{center}
\pause
\begin{center}
\textit{Die physikalischen Gegebenheiten sowie die Ausf"uhrung der gew"unschten Instruktionen ist nicht teil des Message Passing Modells.}
\end{center}

\end{frame}

\section{Die Programmiersprache Erlang}

\begin{frame}

\frametitle{Die Programmiersprache Erlang}

\begin{itemize}
\item funktional
\pause
\item deklarativ
\pause
\item konzipiert f"ur nebenl"aufige, sehr gro"se Programme die im Dauerbetrieb und in Echtzeit in einem verteilten System laufen
\end{itemize}

\begin{flushright}
\cite{Armstrong96erlang}\\
\cite{erl_history}
\end{flushright}

\end{frame}

\begin{frame}

\frametitle{Nebenl"aufigkeit in Erlang}

\begin{itemize}

\item Funktion \textcolor{red}{spawn}(\textit{Module}, \textit{Exported Function}, \textit{List of Arguments})
\pause
\item Konstrukt \textcolor{red}{receive}
\pause
\item Operator \textcolor{red}{!}
\pause
\item Funktion \textcolor{red}{self}()

\end{itemize}

\begin{flushright}
\cite{erl_doc}\\
\cite{erl}
\end{flushright}

\end{frame}

\begin{frame}

\frametitle{Message Passing Modell von Erlang}

\begin{itemize}
\item Verbindung: Erlang Laufzeitsystem stellt Verbindung her (Operator \textcolor{red}{!})\\
\pause
\item Bef"ahigung: Message Queues (Konstrukt \textcolor{red}{receive})
\pause
\item Vollst"andigkeit: Bewerkstelligt das Erlang Laufzeitsystem
\pause
\item Verst"andlichkeit: Erlang arbeitet mit Pattern Matching
\end{itemize}

\end{frame}

\section{Demonstration}

\begin{frame}

\frametitle{Demonstration}

\begin{center}
Diffie-Hellman Schl"usselaustauschalgorithmus implementiert in Erlang\\
mit Hilfe von Message Passing\\
Auch zu finden unter: \textcolor{blue}{https://github.com/c-bebop/message\_passing}
\end{center}

\end{frame}

\begin{frame}

\frametitle{References}

\begin{tiny}

\bibliographystyle{authordate1}
\bibliography{lib}

\end{tiny}

\end{frame}

% etc
\end{document}